\documentclass{article}

\usepackage{blindtext}
\usepackage[utf8]{inputenc}

\usepackage{graphicx,fancyhdr,amsmath,amssymb,amsthm,subfig,url,hyperref}
\usepackage{algorithm}
\usepackage{algpseudocode}
\usepackage{booktabs}
\usepackage{float}
\restylefloat{table}
\usepackage[margin=1in]{geometry}
\usepackage{amsthm}
\usepackage{multirow}
\usepackage{enumitem}
%\usepackage{enumerate}
\usepackage[normalem]{ulem}
\usepackage{accents}
\usepackage{thm-restate}
\usepackage{physics}
\usepackage[numbers]{natbib}

\usepackage{todonotes}

\newcommand{\specialcell}[2][c]{%
  \begin{tabular}[#1]{@{}c@{}}#2\end{tabular}}

%----------------------- Macros and Definitions --------------------------

%%% FILL THIS OUT
\newcommand{\studentname}{Oliver Hinder}
\newcommand{\suid}{ohinder}
%%% END

\newcommand{\ubar}[1]{\underaccent{\bar}{#1}}

 
%\usepackage{algorithm,algcompatible,amssymb,amsmath}

%\renewcommand{\COMMENT}[2][.5\linewidth]{%
%  \leavevmode\hfill\makebox[#1][l]{//~#2}}
%\algnewcommand\algorithmicto{\textbf{to}}
%\algnewcommand\RETURN{\State \textbf{return} }

% If {a} then b End
\algnewcommand{\IfR}[1]{\State\algorithmicif\ #1\ \algorithmicthen}
\algnewcommand{\EndIfR}{\unskip\ \algorithmicend\ \algorithmicif}


\renewcommand{\theenumi}{\bf \Alph{enumi}}

\theoremstyle{plain}
\newtheorem{theorem}{Theorem}
\newtheorem{assumption}{Assumption}
\newtheorem{example}{Example}


\newtheorem{claim}[theorem]{Claim}
\newtheorem{observation}[theorem]{Observation}

\newtheorem{lemma}[theorem]{Lemma}
%\newtheorem{assume}[assume]{Assumption}
%\newtheorem{example}[example]{Example}


\newtheorem{corollary}[theorem]{Corollary}
\newtheorem{definition}[theorem]{Definition}
\newtheorem{remark}[theorem]{Remark}
\newtheorem{fact}[theorem]{Fact}
%\declaretheorem[name=Theorem,numberwithin=section]{thm}
%\declaretheorem[name=Lemma,numberwithin=section]{lem}


\newcommand{\matStart}{\begin{pmatrix}}
\newcommand{\matEnd}{\end{pmatrix}}

\fancypagestyle{plain}{}
\pagestyle{fancy}
\fancyhf{}
\fancyhead[RO,LE]{\sffamily\bfseries\large Stanford University}
\fancyfoot[LO,RE]{\sffamily\bfseries\large \studentname: \suid @stanford.edu}
\fancyfoot[RO,LE]{\sffamily\bfseries\thepage}
\renewcommand{\headrulewidth}{1pt}
\renewcommand{\footrulewidth}{1pt}

\graphicspath{{figures/}}


\def\CompBeta{\beta}

\def\CompBetaAgg{\sigma_{7}}

\def\ArmijoBeta{\sigma_{2}}

\def\ProxResBeta{\sigma_{3}}

\def\BetaSafeAggr{**?**}

\def\BetaSafe{\sigma_{4}}

\def\BetaBacktrack{\sigma_{5}}

\def\BetaVarTheta{\sigma_{6}}

\def\NumCon{m}
\def\NumVar{n}
\def\n{n}
\def\L{L}
\newcommand{\bigO}[1]{O\left( #1 \right)}
\newcommand{\bigSO}[1]{\hat{O}\left( #1 \right)}

\def\R{R}


\def\ResKKT{\mathcal{K}}
\def\ResPrimal{\mathcal{P}}
\def\ResComp{\mathcal{C}}
\def\MeritComp{\mathcal{\zeta}}

\def\T{\mathcal{T}_{\mu}}
\def\Q{\mathcal{Q}_{\mu}}

\def\v{v}

\def\LipP{L_{p}}
\def\LipFun{L_{0}}
\def\LipGrad{L_{1}}
\def\LipHess{L_{2}}
\def\LipCube{L_{3}}

\def\DeltaF{D_{f}}

\def\Diag{\text{Diag}}

\def\X{\mathcal{X}}

\def\Lag{\mathcal{L}}

\def\N{\mathbb{N}}


% ALGORITHMS

\def\AlgTrust{Primal-dual-trust-region}
\newcommand{\callAlgTrust}[1]{\hyperref[AlgTrust]{\Call{\AlgTrust}{#1}}}
\def\AlgMain{non-convex-IPM}
\newcommand{\callAlgMain}[1]{\hyperref[AlgMain]{\Call{\AlgMain}{#1}}}




\begin{document}

\title{Notes on solver}
\author{Oliver Hinder, Yinyu Ye}

\maketitle


\abstract{
Solver
}



\section{Log barrier sub-problems}

This paper is concerned with the following problem:
\begin{subequations}\label{solve-problem}
\begin{flalign}
& \min{f(x) - \mu \log( s )} + \frac{1}{2} d_{x}^T D_{x} d_{x} + \frac{1}{2} d_{s}^T D_{s} d_{s} \\
& a(x)  - s = r \mu \\
& s \ge 0
\end{flalign}
\end{subequations}
The KKT conditions for \eqref{solve-problem} are:
\begin{subequations}\label{KKT-barrier}
\begin{flalign}
\nabla_{x} \Lag(x,y) = \nabla f(x) + D_{x} d_{x}  - \nabla a(x)^T y &= 0 \\
\ResComp_{\mu}(s, y) = Y s - \mu e &= 0  \\
\ResPrimal_{\mu}(x, s) = a(x) - s - \mu r &= 0 \\ 
s, y &\ge 0
\end{flalign}
\end{subequations}
Where the Lagrangian $\Lag(x,y) := f(x) - y^T a(x)$.

We combine the log barrier merit function and the complementary conditions as follows:
\begin{flalign}
\phi(x,y) = \psi(x) + \MeritComp(x,y)
\end{flalign}
With:
$$
\MeritComp(x,y) = \frac{\| \ResComp(x, y) \|_{\infty}^3}{\mu^2}
$$

We now introduces models to locally approximate these merit functions $\nabla_{x} \Lag(x,y)$, $\psi$, $\ResComp$ and $\phi$ respectively. To describe our approximations of a function $f$ around the point $(x, y)$ we use the function $\tilde{\Delta}_{(x,y)}^{f}(u, v)$ to denote the predicted increase in the function $f$ at the new point $(x + u, y + v)$. Observe that we use different approximations depending on the choice of function $f$.


We use a typical linear approximate of $\nabla_{x} \Lag(x,y)$ as follows:
\begin{flalign}
\tilde{\Delta}_{(x,y)}^{\nabla_{x} \Lag} (d_{x}, d_{y}) = \nabla_{x,x} L(x,y) d_{x} + \nabla a(x) d_{y}
\end{flalign}
The following function $\tilde{\Delta}_{(x,y)}^{\psi} ( u )$ is an approximation of the function $\psi(x)$ at the point $(x,y)$ and predicts how much the function $\psi$ changes as we change the current from $x$ to $x + u$.
\begin{flalign}
\tilde{\Delta}_{(x,y)}^{\psi} ( u ) = \frac{1}{2} u^T M(x, y) u + \nabla \psi(x)^T u
\end{flalign}

With:
\begin{flalign}
M (x,y) = \nabla^2 \Lag (x, y) + \sum_i{ \frac{y_i}{a(x)} \nabla a(x)^T \nabla a(x) }
\end{flalign}  
Note that if we set $y_i = \frac{\mu}{s_i}$ then $M(x,y) = \nabla^2 \psi(x)$ and $\tilde{\Delta}_{(x,y)}^{\psi}$ becomes the second order taylor approximation of $\psi$ at the point $x$. Thus we can think of $\tilde{\Delta}_{(x,y)}^{\psi} ( u )$ as a primal-dual approximation of the function $\psi$. 

We can also build a model of the $\MeritComp(x,y) $ as follows:
\begin{flalign}
\tilde{\Delta}^{\MeritComp}_{(x,y)}( d_{x}, d_{y} ) = \frac{\| S y + Y d_{s} + S d_{y} - \mu e \|_{\infty}^3 - \| \ResComp(x, y)  \|_{\infty}^3}{\mu^2}
\end{flalign}
With $S$ a diagonal matrix containing entries of $a(x)$ and $d_{s} = \nabla a(x) d_{x}$. This model $\tilde{\Delta}^{\ResComp}_{(x,y)}$ corresponds to the typical primal-dual linear model of $\ResComp$ i.e. $C(x + d_{x}, y + d_{y}) \approx S y + Y d_{s} + S d_{y} - \mu e$.

With $S$ and $Y$ contain the diagonal elements of $a(x)$ and $y$ respectively.

This allows us to approximate the change in the function $\phi$ at the point $(x,y)$ as follows:
\begin{flalign}
\tilde{\Delta}^{\phi}_{(x,y)}(d_{x}, d_{y}) = \tilde{\Delta}^{\psi}_{(x,y)}( d_{x} ) +   \tilde{\Delta}^{\MeritComp}_{(x,y)}( d_{x}, d_{y} )
\end{flalign}

We say an iterate $(x, y)$ satisfies approximate complementary if $(x,y) \in \Q$ where $\Q$ is defined as follows:
\begin{flalign}\label{approximate-complementary}
\Q = \left\{ (x, y) \in \R^{\NumVar} \times \R^{\NumCon} : a(x) > 0, y > 0, \| \ResComp(x, y) \|_{\infty} \le \frac{\mu}{2} \right\}
 \end{flalign}
We say the point $(x, y)$ is a $\mu$-scaled KKT point if $(x,y) \in \T$ where: 
\begin{flalign}\label{first-order-necessary}
\T = \left\{ (x, y) \in \Q :  \| \nabla \Lag (x, y) \| \le \mu ( \| y \|_1 + 1)  \right\}
 \end{flalign}
 
In which case the algorithm terminates.
 
\section{Algorithm}\label{sec:alg}

Let $S$, $Y$ denote the diagonal matrices with entries of $s$ and $y$ respectively. We can linearize \eqref{KKT-barrier} at the iterate $(x, y, s)$ as follows:
\begin{flalign}
\begin{bmatrix}
\nabla^2 \Lag (\hat{x}, \hat{y}) + D_{x} &  -\nabla a(\hat{x})^T & 0 \\
\nabla a(\hat{x}) & 0 & -I \\
0 & \hat{S} & \hat{Y} + D_{s}
\end{bmatrix} 
\begin{bmatrix}
d_x \\
d_y \\
d_s
\end{bmatrix}
&= -\begin{bmatrix}
\nabla \Lag(x, y) \\
\ResPrimal_{\mu}(x, s)  \\
\ResComp_{\mu}(s, y)
\end{bmatrix}
\end{flalign}

Which is equivalent to solving:

\begin{flalign}
\begin{bmatrix}
\nabla^2 \Lag (\hat{x}, \hat{y}) + \nabla a(x)^T D_{s} \nabla a(x) + D_{x}  &  \nabla a(\hat{x})^T  \\
\nabla a(\hat{x}) & -(\hat{Y}  + D_{s})^{-1}  \hat{S}  \\
\end{bmatrix} 
\begin{bmatrix}
d_x \\
-d_y 
\end{bmatrix}
&= -\begin{bmatrix}
\nabla \Lag(x, y) \\
\ResPrimal_{\mu}(x, s)  + (\hat{Y} + D_{s})^{-1} \ResComp_{\mu}(s, y)
\end{bmatrix}
\end{flalign}

One can also solve this system by solving the Schur complement:
$$
(\nabla^2 \Lag (\hat{x}, \hat{y}) + \nabla a(\hat{x})^T (\hat{Y}  + D_{s}) \hat{S}^{-1} \nabla a(\hat{x})  + D_{x} ) d_{x}  = -\nabla \Lag(x, \mu S^{-1} e) - \nabla a(\hat{x})^T  \hat{Y} \hat{S}^{-1} \ResPrimal_{\mu}(x, s) 
$$



Observe that \eqref{newton-system} may be singular or correspond to a direction that makes the log barrier objective worse. To rectify this problem we compute the direction as follows:
\begin{subequations}\label{compute-directions}
\begin{flalign}
& d_x = \arg \min_{\| u \|_{2} \le r}{ \tilde{\Delta}^{\psi}_{(x,y)} (u) } \label{compute-directions-dx} \\
& d_s = \nabla a(x) d_{x} \\
& d_y =  -S^{-1} \left( Y d_{s} + \ResComp (x, y) \right) \label{compute-dy}
\end{flalign}
\end{subequations}

******CAREFUL WITH SIGNS i.e. should be $d_{s} = - \nabla a(x) d_{x}$,  $d_{y} = -S^{-1} \left( Y d_{s} + \ResComp (x, y) \right)$ ***************

It is well-known from trust region literature that there exists some $\delta \in [0, \infty)$ such that:
\begin{flalign}\label{trust-region-linear-system}
(M(x,y) + \delta I) d_x = -\nabla \psi(x)
\end{flalign}
Furthermore, by re-arranging this equation we can deduce that $(d_x, d_y, d_s)$ satisfies a perturbed version of \eqref{newton-system}:
\begin{flalign}\label{perturbed-newton-system}
\begin{bmatrix}
\nabla^2 \Lag (x, y) + \delta I & -\nabla a(x)^T & 0  \\
-\nabla a(x) & 0 & I \\
0 & S & Y
\end{bmatrix} 
\begin{bmatrix}
d_x \\
d_y \\
d_s
\end{bmatrix}
&=  -\begin{bmatrix}
\nabla \Lag(x, y) \\
0 \\
\ResComp(x, y)
\end{bmatrix}
\end{flalign}

\begin{algorithm}[H]
\caption{Primal-dual trust region step}\label{AlgTrust}
\begin{algorithmic}
\Function{\AlgTrust}{$x, y, r$}
\begin{subequations}\label{compute-directions}
**** $\in$ ****
\begin{flalign}
& d_x \in \arg \min_{\| u \| \le r}{ \tilde{\Delta}^{\psi}_{(x, y)} ( u ) } \label{compute-directions-dx} \\
& d_s = \nabla a(x) d_{x} \\
& S = \Diag(a(x)) \\
& d_y =  - S^{-1} \left( Y d_{s} + \ResComp (x, y) \right)
\end{flalign}
\end{subequations}
\State $(x^{+}, y^{+}) \gets (x + d_{x}, y + d_{y})$
\State $\Return (x^{+}, y^{+}, d_{x}, d_{y})$
\EndFunction
\end{algorithmic}
\end{algorithm}

Our complete algorithm is summarized as follows:

\begin{algorithm}[H]
\begin{algorithmic}\label{AlgMain}
\Function{\AlgMain}{$x^1, y^1$}
\For{$k = 1, ... , \infty$}
\State $r \gets R (y^k)$
\Repeat
\State $(x^{+}, y^{+}, d_{x}, d_{y}) \gets \callAlgTrust{x^k, y^k, r}$
\If{$(x^{+}, y^{+}) \in \Q$}
\If{$(x^{+}, y^{+}) \in \T$}
\State \Return{$(x^{+}, y^{+})$}
\EndIf
\EndIf
\State $r \gets r / 2$
\Until{$\phi(x^{+}) > \phi(x^k) + \frac{1}{2} \tilde{\Delta}_{(x^k,y^k)}^{\phi}  ( d_{x}, d_{y} )$}
\State $x^{k} \gets x^{+}$
\State $y^k \gets y^{+}$
\EndFor
\EndFunction
\end{algorithmic}
\caption{Primal-dual non-convex interior point algorithm}
\end{algorithm}


\section{Delta computation}

\begin{algorithm}[H]
\begin{algorithmic}
\State $\lambda_{lb} = 0$, $\lambda_{ub} = \delta_{\max} = \| H \|^2_{F}$, $\delta_{k-1}$ \Comment{lower and upper bounds on minimum eigenvalue}
\State Try $\delta = 0$, if succeeds, trial solve with this delta. If step size is small skip to trust region step.
\State $\delta = \delta_{k-1}$
 \If{$\delta = 0$}
\State $\delta = \delta_{\min}$
\EndIf
\For{$i = 1, ..., \infty$}
\State Break if inertia correct and update $\lambda_{lb}$ and $\lambda_{ub}$.
\State $\delta = \delta 100$
\EndFor
\State \text{Trust region}
\State $R = \| d_{x}^{k-1} \|_{2}$
\For{$i = 1, ..., \infty$}
\State Compute trust region with $R$
\State If trust region is too accurate increase radius size
\State If step unsuccessful decrease radius size
\State Prevent oscillation
\EndFor
\end{algorithmic}
\caption{Delta}
\end{algorithm}

\end{document}